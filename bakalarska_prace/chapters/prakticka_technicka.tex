\chapter{Praktická část - návrh aplikace a technické řešení}

% \section{Uživatelské workflow a use-cases}
% TODO?

\section{Technické řešení}
\subsection{Výběr technologií}
Prototyp byl vytvořen jako webová aplikace.
Hlavní důvody pro toto rozhodnutí jsou:
\begin{itemize}
	\item dostupnost pro uživatele i testery\\
	Není potřeba nic instalovat, stačí jakýkoli moderní webový prohlížeč.
	\item snadné vytvoření prototypu
	\item snadná rozšiřitelnost a kooperace na projektu\\ Jako open-source projekt předpokládá možnosti další spolupráce, například s~designéry. Úprava grafické stránky aplikace je v~případě webové aplikace čistě HTML+CSS(+JS), což je rozšířená dovednost, oproti jiným grafickým prostředím (jako Qt či herní enginy).
\end{itemize}

Konkrétně byl zvolen webový framework \textbf{Flask} (založeného na jazyce Python). Motivací byla osobní zkušenost s~tímto frameworkem, a tedy dostatečná představa o~realizovatelnosti tohoto typu aplikace v~daném prostředí.

Na frontendové straně nebyl použit žádný rozsáhlý framework a pro někde nutné části Javascriptu byly zvoleny pouze microframeworky \textbf{stimulus.js} a \textbf{jQuery}.

Vývoj je verzován systémem \textbf{git} za použití veřejného repozitáře na službě GitHub.

\subsection{Základní funkcionality}
\subsubsection*{Nahrávání předem definovaných dat Mise}
	\textbf{Východiska}\\
	Mise musí být schopné používat předem definovaná data. Tato data mohou být v~různé formě.\\
	V~budoucnu by aplikace mohla nabízet možnosti, jak přidávat další data misí.
	
	\textbf{Řešení}\\
	Zvažováno bylo několik řešení:
	
	\begin{itemize}
		\item data v~databázi -- například SQLite (která může být oproti jiným přímo součástí projektu)
		\item data přímo v~kódu
		\item data v~externím strukturovaném souboru
	\end{itemize}

	Varianta databáze byla pro účely prototypu vyřadazena z~důvodu složitější úpravy a nahlížení dat. \\
	Z~dlouhodobého hlediska se jeví jako vhodná forma strukturovaného dokumentu (např. JSON) -- má výhody ve snadném přidávání dalších scénářů a jasného oddělení dat a funkcionality (kódu). Vyžaduje však vytvoření kódu pro převod těchto dat do objektové struktury, kterou používá samotná aplikace.
	Proto byla tato varianta zařazena pro prototyp jako \textit{nice-to-have}, a v~první verzi prototypu jsou data přímo v~kódu v~kontroleru mise.

	V~případě velkého objemu dat by forma nahrávání externího souboru nemusela být nejvhodnější, a bude možná nutné přejít na variantu s~databází, či jinou. Díky použití ORM knihovny SQLAlchemy, a abstrahování logiky do modelové/objektové vrstvy to ovšem nebude znamenat velký zásah do kódu. 

\subsubsection*{Zobrazení mise s~jednotlivými záložkami}
	\textbf{Východiska}\\
	V~náhledu mise je třeba uživateli zobrazit následující obsah:
	
	\begin{itemize}
		\item Úvod -- představení mise
		\item Data -- několik různých typů dat
		\item Místo pro řešení -- stránku nebo stránky, kde uživatel řeší daný úkol - zadává informaci (např. heslo), vybírá z~uvedených možností a podobně
		\item Závěr -- stránku s~přehledem použitých zdrojů a doplňujících informací. 
	\end{itemize}	

	\textbf{Řešení}\\
	V~prototypu je zobrazování vyřešeno nahráním všech datových zdrojů při načtení stránky a zobrazování/skrývání pomocí jednoduchého javascriptového (stimulus.js) kontroleru. Kontroler funguje obecně pro libovolný počet datových zdrojů.
	Toto řešení má oproti jiným variantám jednoduchý kód (přehledná šablona a malý přehledný javascriptový kontroler).
	Potenciální riziko je v~tom, že pokud budou datové zdroje rozsáhlejší, bude první nahrání stránky pomalé. V~takové situaci by bylo vhodné data nenahrávat všechna při prvotním načtení stránky, ale přidat asynchronní získání dat pomocí Fetch API. To se však nijak nevylučuje s~vytvořeným kontrolerem, který má na starost pouze přepínání viditelnosti jednotlivých částí stránky.

\subsubsection*{Zobrazovat data různých typů}
	\textbf{Východiska}\\
	Na základě dat z~kapitoly \textit{Hlavní zdroje dat digitální stopy uživatele} byly definovány základní typy dat, které se v~náhledech misí mohou objevovat:

	\begin{itemize}
		\item Příspěvky na sociálních sítích
		\item Výměna zpráv mezi dvěma osobami
		\item Výpis dat\\
			seznam hovorů, výpis z~bankovního účtu,...
		\item Datové body na mapě
		\item Souhrnné informace\\
			například informace, které uchovávají sociální sítě o~jednotlivých uživatelích
		\item Webové stránky
	\end{itemize}

	Kromě toho ale lze očekávat, že se budou přidávány další typy dat, je tedy důležité, aby bylo snadné tuto nabídku rozšiřovat,a aby bylo řešení dostatečně obecné (nebo zobecnitelné), aby bylo toto přidávání snadné. 

	\textbf{Řešení}\\
	Pro jednotlivé typy dat byly vytvořeny samostatné modely, kontrolery a sady šablon. Pro ukázku je zde struktura kódu pro zobrazování dat typu sociální síť:\\
	Kontroler umožňuje zobrazovat několik základních stránek - osobní profil, sadu příspěvků (feed) a přihlašovací stránku.
	Zobrazování sady příspěvků je realizováno pomocí několik \textit{partial} šablon - feed, post, comment. Jednotlivé šablony je pak snadné například designově upravovat a zůstávají velice přehledné.

	U~většiny typů dat bylo třeba si vytvořit vlastní HTML+CSS kód, který nabízí očekávanou strukturu dat v~přehledné podobě.
	Pro zobrazování lokačních dat byla využita javascriptová knihovna SMapy.   


\subsection{Další funkcionality}
Mezi funkcionality, které nejsou potřebné pro fázi prototypu, ale je dobré na ně myslet při strukturování celé aplikace, patří:

\begin{itemize}
	\item \textbf{Uživatelský účet}\\
		Aplikace by měla nabízet uchovávání informací o~stavu jednotlivých misí, aby mohl uživatel navázat tam, kde přestal.
		Může to být řešeno uživatelským účtem s~registrací, nebo identifikací pomocí cookies.

	\item \textbf{Přidávání dalších scénářů}\\
		Aplikace do budoucna počítá s~možností vlastních scénářů. Systém zadávání zdrojových dat a jejich nahrávání tedy musí být možné vystavit ven.
		Varianty přidávání můžou být různě technicky složité, to bude záležet na další analýze používání aplikace v~praxi.		


\end{itemize} 

\subsection{Návrh aplikace}
Aplikace má architekturu MVT (Model-View-Template), inspirovanou návrhem jiného webového frameworku v~jazyce Python, a to Django.

\subsubsection*{Model}
Aplikace plně využívá objektový návrh a se všemi zdrojovými daty nakládá jako s~objekty.\\
Jsou tedy definované modely/třídy jako \verb|Facebook user|, \verb|Facebook post|, \verb|Location point| a další.\\
Kromě toho je objektem i každá Mise, která obsahuje jednotlivé \verb|Mission Items|.

\subsubsection*{View}
Vrstva View řeší získávání a zpracování dat, vykreslení šablony s~těmito daty a navázání na konkrétní cesty (\textit{route}).\\
V~aplikaci je použito rozšíření \verb|Flask-Classful|, které usnadňuje práci s~views, například snadným vytvořením základních \verb|CRUD| operací.

\subsubsection*{Template}
Flask v~základu využívá šablonovací jazyk \verb|Jinja2|, který nabízí omezené množství logiky v~šabloně, a vede k~tomu, aby většina aplikační logiky zůstávala mimo šablonu (tedy ve View).

\subsection{Ověření a otestování aplikace}
Při ověřování funkčnosti aplikace s~reálnými uživateli budou zejména kontrolovány tyto oblasti: 

\begin{itemize}
	\item \textbf{Orientace v~prostředí}\\
		Bude ověřováno, zda je uživateli dostatečně jasné, co se po něm požaduje, a dokáže v~prostředí aplikace najít vše, co potřebuje pro interakci s~ním. 
	\item \textbf{Používání datových zdrojů}\\
		S~ohledem na to, že nahlížení datových zdrojů je v~uživatelské interakci s~aplikací hlavní činností, bude ověřováno, zda je jejich používání jasné, srozumitelné a umožňuje plnění misí. 
	\item \textbf{Náročnost úkolů}\\
		Další testovanou oblastí bude náročnost úkolů v~prezentovaných misích. Toto testování musí probíhat u~každé mise, jež bude do aplikace přidána. Bude ověřováno, zda dokáže uživatel z~dodaných informací vymyslet postup práce, zda má všechna potřebná data a naopak jestli není úkol příliš jednoduchý.
	\item \textbf{Míra zaujetí}\\
		Nakonec bude získán pohled uživatelů na to, zda je tato forma seznámení s~tématikou poutavá a přínosná.
\end{itemize}

Ze všech kontrolovaných oblastí lze očekávat výstupy, které poslouží k~posunutí aplikace z~fáze prototypu do produkční verze.    


% \subsection{Vyhodnocení provedeného ověření a doporučení pro další rozvoj aplikace}
% TODO?