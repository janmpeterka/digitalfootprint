\chapter{Hlavní zdroje dat digitální~stopy\\ uživatele}

Tato kapitola nabízí možnou kategorizaci dat podle toho, jak se uživatel podílel na jejich vytvoření, a podle jejich anonymity. Dále rozebírá konkrétní typy dat pro úplnější představu o~tom, co všechno je ukládáno a zpracováváno.\\
Tyto kategorie dat budou využity v~praktické části práce pro vytváření scénářů a modelových dat podle reálných typů dat a možnostech přístupu k~nim.

\section{Kategorie dat}
Data, která vytváříme používáním technologií a digitálních služeb, lze kategorizovat několik způsoby. Dělení může být například na:
\begin{itemize}
	\item \textbf{aktivní} vs \textbf{pasivní} digitální stopu \citep{pew-digital-footprint}
    \item přechozí dělení doplněno o~\textbf{vědomě nevědomou} digitální stopu \citep{fish-digital-footprint}
	\item \textbf{identifikovatelná} vs \textbf{anonymní} data
\end{itemize}


\subsection{Aktivní, pasivní a vědomě nevědomá digitální stopa}
Toto dělení se dívá na data z~pohledu uživatele, vědomí o~jejich existenci a možností vlivu. 

\subsubsection*{Aktivní stopa}
Aktivní stopou se myslí takový typ dat, který uživatel vědomě publikuje. Může jít o~příspěvky, komentáře a reakce na sociálních sitích či osobních webech, fotky a jiné soubory nahrané na cloudová úložiště nebo vytvořené uživatelské účty \citep{pew-digital-footprint}.

V~tomto pohledu nerozlišujeme, kdo je majitelem dat a jak s~nimi kdo může nakládat, to závisí na podmínkách konkrétní platformy (omezené legislativou dané země).

\subsubsection*{Pasivní stopa}

Pasivní stopa se skládá z~typu dat, která uživatel vytváří svým používáním digitální platformy či služby bez přímého sdílení či nahrávání. Nemusí si tedy vzniku těchto dat být vůbec vědom, zejména pokud má nedostatečnou představu o~technologické stránce používaných služeb. Obvykle jde o~analytická data, používaná pro lepší technický, bezpečnostní nebo marketingo-ekonomický efekt. Může jít o~informace o~zobrazení stránky či příspěvku, IP adresu a další technické parametry připojeného uživatele (například lokaci). Může jít i o~kombinovaná data, například určení zájmů či demografické skupiny, vytvořená na základě jednotlivých dat.

\subsubsection*{Vědomě nevědomá stopa}

Fish dále přidává kategorii vědomě nevědomé digitální stopy, která se skládá z~dat aktivně vložených jinými uživateli \citep{fish-digital-footprint}.
Může jít o~fotografie -- na sociálních sítích, ale například i z~různých akcí (kde člověk zveřejnění dat musí v~České republice odsouhlasit), označení v~příspěvcích nebo o~data zveřejněná úřady. U~části těchto dat může mít uživatel možnost tuto stopu zpětně omezit (například Facebook umožňuje odstranění označení).

\subsection{Identifikovatelná a anonymní data}
Toto dělení se dívá na data z~pohledu právně-technologického.

\subsubsection*{Identifikovatelná data}

Sem řadíme typ dat, která jsou přímo propojitelná s~naší osobou. Jde pak o~osobní údaje podle definice GDPR:

\begin{displayquote}
	\uv{Pro účely tohoto nařízení se rozumí „osobními údaji“ veškeré informace o~identifikované nebo identifikovatelné fyzické osobě (dále jen „subjekt údajů“);
	identifikovatelnou fyzickou osobou je fyzická osoba, kterou lze přímo či nepřímo identifikovat, zejména odkazem
	na určitý identifikátor, například jméno, identifikační číslo, lokační údaje, síťový identifikátor nebo na jeden či více
	zvláštních prvků fyzické, fyziologické, genetické, psychické, ekonomické, kulturní nebo společenské identity této
	fyzické osoby}
	\citep{gdpr}
\end{displayquote}

Jde tedy o~data, jež lze nějakým způsobem spojit s~konkrétní osobou. S~ohledem na rychlý vývoj v~oblasti získávání dat, a zároveň malou mírou transparentnosti těchto procesů, je u~dat těžké obecně říct, zda jsou identifikovatelná. Jak ukazuje dále příklad lokačních dat, i data, která prošla procesem anonymizace, mohou být ve skutečnosti identifikovatelná.  

\subsubsection*{Anonymní data}

Může jít o~data anonymizovaná službou či mezivrstvou, nebo také o~data vytvořená uživatelem s~použitím anonymizačních nástrojů, jako je například VPN (virtuální privátní síť, používána pro skrytí IP adresy uživatele).

% TODO - ještě rozepsat

\section{Konkrétní typy dat}

\subsection{Historie prohlížení}
Jednou z~významných součástí pasivní digitální stopy je historie prohlížení -- tedy záznam navštívených webových stránek (případně webových requestů).


Samostatně jde o~data anonymní (nejsou přímo spojená s~naší osobou), ovšem v~realitě to tak nemusí být.

Tato data může uchovávat internetový prohlížeč, a to buď lokálně, nebo na cloudu. Například v~případě prohlížeče Google Chrome se zapnutou synchronizací jsou tyto informace ukládány jako součást dat Google profilu.

Data o~webové aktivitě má také poskytovatel internetového připojení (Internet Service Provider -- ISP), který je zároveň má spojená s~naší IP adresou, která je vázaná na konkrétní smlouvu o~poskytování internetu. V~České republice si tato data může vyžádat policie a prokazatelně to dělá \citep{policie-isp}.

Kromě toho je možné aktivitu uživatelů napříč weby sledovat pomocí cookies třetích stran a různých typů fingerprintingu.

\subsubsection*{Cookies}
Jak bylo zmíněno v~předchozí kapitole, cookies jsou soubory, které si stránka prostřednictvním prohlížeče ukládá do počítače uživatele, aby ho mohla identifikovat při dalších požadavcích (zde mluvíme o~\textit{first-party cookies}). V~současnosti mnoho webů obsahuje takzvané cookies třetích stran \textit{third-party cookies}, které umožňují reklamním službám (například Google Ads) sledovat aktivitu napříč webem.
Google v~březnu 2021 oznámil postupný odchod od používání cookies třetích stran, které plánuje nahradit vlastní technologí a ve svém prohlížeči Google Chrome je zakázat (a tím následovat další prohlížeče jako Firefox či Safari) \citep{google-privacy-announcement}.

\subsubsection*{Device a browser fingerprinting}
S~postupným legislativním tlakem na omezení rozsahu cookies se začaly služby přesouvat k~používání \textit{fingerprintingu}, tedy používání jakéhosi otisku zařízení nebo prohlížeče, ze kterého uživatel k~webům přistupuje. Používané techniky jsou rozmanité, od získávání informací o~prohlížeči a operačním systému (verze, jazyk, instalované doplňky a další), po \textit{canvas fingerprint} využívající specifika v~renderování webového prvku \verb|canvas|, které se liší podle GPU nebo grafických ovladačů v~daném zařízení.  

Zjistit svůj browser fingerprint lze například pomocí služby \href{https://amiunique.org/fp}{Am I~Unique}. Služba ukazuje, kolik informací je prohlížeč schopen získat, a to i bez schválení uživatelem.

\subsubsection*{Behavioral profiling}
Relativně nově používanou technikou je \textit{behavioral profiling}. Tato technika se snaží rozlišit uživatele podle jejich chování -- primárně charakteristik pohybu myši nebo způsobu psaní na klávesnici \citep{behavioral-profiling, mouse-behavioral-biometrics, digital-behavior-fingerprint}.\\
Tento způsob je stále ještě méně spolehlivý než dříve zmíněné, neboť využívá výrazně rozmanitější typ dat.
V~kombinaci s~jinými nástroji a s~rostoucími možnosti vyhodnocování těchto dat (například využitím strojového učení) se však stává nezanedbatelnou možností v~repertoáru nástrojů pro profilování a identifikaci uživatele.

\subsection{Lokační data}
Jak bylo zmíněno v~předchozí kapitole, jedním z~typů využívaných dat jsou data lokační.

Tato data mohou například poskytnout vhled, do jakých obchodů uživatel chodí, a nabídnout tak remarketing napříč fyzickým a digitálním světem.

Lokační data je v~současnostni snadné získat, zejména díky množství mobilních zařízení připojených k~internetu. I~aplikace, které reálně data pro svoje funkce nepotřebují, o~ně mohou uživatele požádat. Prodejem těchto dat pak mohou získat finance na samotný vývoj aplikace \uv{zdarma}.\\
Je na místě říct, že možnosti prodeje dat se mohou výrazně lišit podle toho, v~jaké zemi -- a tedy pod jakou legislativou -- se uživatel pohybuje.

Lokační data dále potenciálně sdílejí aplikace, které reálně data používají pro svoje služby -- mapy (\textit{Google Maps}, \textit{Mapy.cz} a podobné), dopravy (\textit{IDOS}, \textit{Pubtran}, \textit{PID Lítačka} a podobné) a dovozy (\textit{DámeJídlo}, \textit{Uber}  a podobné).

Například Google Maps uchovává lokaci a přiřazuje ji ke konkrétním místům (obchody, úřady a podobně). Zároveň uchovává informaci o~předpokládané formě aktivity (např. chůze, běh, kolo, auto, hromadná doprava) včetně důvěry (pravděpodobnosti) v~predikci.

Mohlo by se zdát, že lokační data jsou anonymní (typicky obsahují nějaké náhodné ID zařízení, časovou značku a lokaci), v~realitě jsou však velmi snadno deanonymizována. Profesor Ohm z~Georgetown University říká, že dostatečně přesná geolokační data je naprosto nemožné anonymizovat. Jediná méně anonymizovatelná data jsou podle něj genetická data \citep{location-data}.

\subsection{Data vkládaná na sociální sítě a další platformy}
Nejvýznamnější složkou aktivní (veřejné) digitální stopy jsou pravděpodobně v~současné době obsahy sociálních sítí.
Jde o~příspěvky, ale i komentáře, reakce (emotikony), projevení zájmu o~událost a další vědomé interakce, jejichž výstupem je informace sdílená s~dalšími uživateli. 

Kromě toho, že jsou součástí celkového \uv{balíku} dat, které služby mají, může být jejich riziko i v~nastavení, kdo se k~datům může dostat.
Kvůli nevědomosti uživatele nebo chybou dané služby -- například v~roce 2018 byly příspěvky přibližně 14 milionů uživatelů veřejné místo soukromých \citep{facebook-public-posts} -- se může stát, že jsou data veřejná pro jinou skupinu lidí, než si uživatel myslí. 

A~zároveň, jak již bylo uvedeno v~první kapitole, jsou to data dobře využitelná do modelů, které dokáží o~uživateli odvodit i informace, které vědomě nepíše, a které jsou mnohdy velmi citlivé.

\subsubsection*{Fotografie}
Specifickou částí zveřejňovaných dat jsou fotografie. U~nich může často docházet ke vzniku \textit{vědomě nevědomé} digitální stopy, když se člověk objevuje na fotkách, které sdílí někdo jiný (a naopak, pokud sdílí fotografie, na kterých je někdo další).
Pokrok ve strojové analýze obrazu už umožňuje, aby byly osoby na fotografii rozpoznány automaticky (Facebook díky tomu může u~nahrané fotky navrhnout osoby k~označení, Google Photos umožňuje zobrazit všechny fotografie dané osoby s~velmi vysokou přesností).
Zároveň je často možné z~fotografie získat další informace z~\textit{metadat}. To jsou data, která vytváří fotoaparát (dnes obvykle smartphone) a můžou obsahovat informace jako čas a datum, lokaci, název zařízení nebo technické parametry jako délku expozice či ISO. Některé služby (například Facebook) při nahrávání fotografie citlivá metadata smaže, aby nebylo možné uložením fotografie získat i je, ale uživatel s~tím při nahrávání fotografie obecně počítat nemůže.
Kromě toho se s~rozvojem strojového učení ukazuje, že bude nejspíš možné lokaci často velice přesně určit i bez lokačních metadat, a to až s~přesností ulice či domu \citep{ai-photo-location}.

\subsection{Další data sledovaná sociálními sítěmi a podobnými platformami}
Kromě dat, která na sociální sítě uživatel vědomě vkládá, tyto služby sbírají a vytváří další data, napojená na jeho profil. Alespoň část je možné vidět vyžádáním těchto dat od dané služby. Toto právo v~EU zajišťuje GDPR.\\
V~datech z~Facebooku je možné vidět typy dat, o~kterých uživatel obvykle neví. Tato data nejsou nikde ve službě přímo zobrazována a nejsou třeba pro uživatelskou funkcionalitu služby:
\begin{itemize}
	\item \textbf{friend\_peer\_group}\\
	Facebook si na základě dat o~uživateli a jeho vazbách na další uživatele jeho profil zařadí do některé z~předem definovaných skupin, například \textit{Začínající dospělý život}, a tu pak využívá pro cílení reklamy a celkové nastavení toho, co uživatel vidí.
	
	\item \textbf{viewed}\\
	Facebook sleduje ve speciální kategorii
	\begin{itemize}
		\item jaká videa, jakou část z~nich, a kolik času celkově uživatel strávil u~videí z~Facebook View;
		\item jaké zboží na Facebook Marketplace uživatel prohlížel;
		\item na jaké zobrazené reklamy reagoval.
	\end{itemize}
	Celkově zde jde o~data, která jsou velmi cenná z~pohledu reklam, monetizace a udržení uživatele ve službě.
	
	\item \textbf{ads\_interests}\\
	Facebook k~uživateli přiřazuje zájmy, které jsou pak využívány na reklamních aukcích.

	\item \textbf{unfollowed\_pages}\\
	Facebook zachovává historii toho, jaké stránky uživatel přestal sledovat.

	\item \textbf{removed\_friends}\\
	Stejně tak uchovává informaci o~ukončených \uv{přátelstvích}.

	\item \textbf{group interactions}\\
	Počet interakcí (příspěvků, komentářů, reakcí) ve skupinách, jejichž je uživatel členem.

	\item \textbf{people interactions}\\
	Počet interakcí s~osobními profily.

	\item \textbf{your\_topics}\\
	Podobně jak ads\_interests -- seznam témat, jež se Facebook domnívá, že uživatele zajímají, a podle nich přizpusobuje zobrazovaný obsah.
\end{itemize}

Podobně je možné se podívat na některá data, která vytváří a uchovává Google (samozřejmě podle toho, které služby uživatel používá). Zde jsou uvedena některá překvapivější, která se nezdají nutná pro základní funkcionalitu (v~této sekci jsou uvedena data, která nebyla zařazena do jiné sekce této kapitoly).
U~každého bodu je pak uvedeno (pokud to je možné zjistit), skrze jakou službu tato data vznikají:

\begin{itemize}
	\item \textbf{názvy tisknutých souborů}\\
	Služba \textit{Google Cloud Print}, umožňující tisk přes internet, uchovává seznam s~názvem tisknutých souborů.

	\item \textbf{informace o~hrách}\\
	O~uživatelích služby \textit{Google Play Games} jsou uchovávána například data o~tom, kdy poprvé a kdy naposledy hráli danou hru, a souhrnné herní statistiky. Kromě toho jsou v~jiné sekci uloženy všechny záznamy spuštění her. 

	\item \textbf{hudba}\\
	Služba \textit{Google Play Music} uchovává mimo jiné informaci o~počtu přehrání jednotlivých skladeb, a seznam všech spuštění nahrávek s~jejich časovou značkou. To platí i u~služby \textit{Google Podcasts}.

	\item \textbf{aplikace}\\
	Služba \textit{Google Store} -- tedy nejpoužívanější služba pro správu a instalaci aplikací na zařízeních s~operačním systémem \textit{Android} -- kromě seznamu instalovaných aplikací zaznamenává údaje o~zařízení (konkrétní model telefonu), na kterých byla instalována, a telefonního operátora. Dále je uložena informace o~každé instalaci a odinstalaci aplikací.
	
	\item \textbf{reklamy}\\
	Google uchovává informaci o~všech reakcích (kliknutích) uživatele na zobrazené reklamy.

	\item \textbf{Android}\\
	Google uchovává seznam všech otevření aplikací. Vztahuje se na uživatele používající operační systém \textit{Android} s~přihlášeným profilem Google. 

	\item \textbf{hlasové pokyny}\\
	Hlasové pokyny pro službu \textit{Google Assistant} jsou ukládány jako originální zvukové nahrávky. To se děje i v~případech, kdy začalo zařízení poslouchat \uv{omylem} -- kdy uživatel neřekl klíčové slovo pro začátek poslechu, ale jiný zvuk tak byl mylně interpretován. 

	\item \textbf{vyhledávání}\\
	Služba \textit{Google Search} uchovává informaci o~každém vyhledávání a o~tom, kam z~něj uživatel pokračoval.
\end{itemize}

\subsection{E-mail}
Velká část online komunikace se stále děje přes e-mailové služby, zejména v~případě komunikace pracovní. Ta může obsahovat mnoho (zejména pro firmu) citlivých informací, a jejich únik může mít velké následky (jak je vidět na příkladu uniklých e-mailů Hillary Clinton v~prezidentské volbě v~USA v~roce 2016).

Kromě samotného obsahu e-mailu se dnes používají technologie, které umožňují sledovat uživatelovu interakci s~přijatým e-mailem. Samotná technologie e-mailu toto sledování nenabízí (oproti různým chatovacím službám, nebo protokolu RCS, postupně nahrazujícímu SMS). Často používanou technikou je \textit{pixel tracking}, který funguje pomocí vložení \uv{neviditelného} obrázku o~velikosti jednoho pixelu, který se při otevření e-mailu nahrává z~unikátní adresy, a tento dotaz na danou adresu je možné zaznamenat. O~tom, že je otevření e-mailu sledováno, obvykle uživatel vůbec neví.

\subsection{Finanční záznamy}
V~době, kdy velká většina finančních transakcí neprobíhá s~fyzickými penězi (zejména v~České republice, která je k~adopci nových technologií v~oblasti převodu peněz velmi otevřená), se vytváří další část digitální stopy právě v~této oblasti.
Primárně mají veškeré záznamy k~dispozici banky, u~kterých transakce probíhají. Tyto informace podléhají bankovnímu tajemství, může si je však v~oprávněných případech (dle § 8 zákona č. 141/1961 Sb., trestní řád) vyžádat policie, a to bez vědomí osoby, jejíž informace jsou nahlíženy. Výpisy z~účtu můžou být vyžadovány i v~jiných případech, třeba při bezpečnostní prověrce.

Znatelná část finančních transakcí se rovněž přesouvá do plateb telefonem, které probíhají přes dalšího prostředníka -- data o~plabách tedy nemá již jen banka, ale i další soukromá firma. Často používanými nástroji pro tyto platby jsou Alipay a WeChat (obojí primárně v~Číně), Apple Pay, PayPal, Samsung Pay, Amazon Pay nebo Google Pay. Většinou tedy jde o~firmy, které poskytují i další služby a historie transakcí pak může být dalším zdrojem dat o~uživateli.

\subsection{Zdravotní data}
Další oblastí velmi citlivých dat jsou data zdravotní. S~postupnou digitalizací zdravotnictví bude možné tato data čím dál více propojovat, což nese nesporné výhody ve snížení zdravotnické administrativy a urychlení péče, ale i potenciální rizika v~případě úniku či zneužití těchto dat.

Kromě dat ve zdravotnictví se čím dál rozmanitější typy dat sbírají skrze \textit{wearables} -- nositelnou elektroniku, často nabízející různé fitness a zdravotní funkce. Aktuálně se tedy můžeme setkat se sběrem dat o~pohybu, měřením tepu a dechu, nebo měřením odporu kůže či množství kyslíku v~krvi.
Tato data jsou pak automaticky zpracovávána pro rozpoznávání typu a intenzity pohybu či sportu, doby a kvality spánku nebo míru stresu.

Dá se očekávat, že se budou objevovat další typy sledovaných dat, sensory budou přesnější a z~dat (a to i historických) bude možné vyčíst více informací -- probíhají mnohé výzkumy, které se snaží v~datech najít návaznost na různé zdravotní jevy, příkladem může být studie propojující informace o~variabilitě tepu s~psychickým i fyzickým zdravím \citep{heart-rate-health}.

Tato data typicky opět sbírají soukromé firmy, často nabízející mnoho dalších služeb -- uvést můžeme Google (který v~roce 2019 koupil firmu Fitbit, jednoho z~předních výrobců fitness zařízení), Apple či Samsung.

Ještě výrazně problematičtější částí problematiky zdravotních dat může být v~posledních letech velmi populární analýza DNA soukromými firmami. DNA je ze své podstaty extrémě osobní informace. Ani firmy poskytující tento typ služeb (a spravující genetická data, která jim klienti poskytují k~dalšímu využití) nejsou odolné vůči únikům dat, jak je vidět na případu firmy MyHeritage, které v~roce 2018 unikla data 92 milionů účtů \citep{myheritage-leak}. V~tomto případě nedošlo k~úniku genetických dat, to je ovšem pravděpodobně jen otázka času. Aktuálně jsou genetická data užitečná zejména pro výzkum v~oblasti medicíny a léčiv, s~rostoucím pochopením genomu je těžké odhadnout, jakými způsoby by mohla být tato data využita či zneužita.

% \subsection{Státní rejstříky a databáze}
% TODO?

\section*{Shrnutí}
V~této kapitole byly popsány možné způsoby kategorizace osobních dat a ukázány některé běžné typy osobních dat.
Zároveň je u~jednotlivých kategorií dat ukázáno, jaká jsou rizika s~nimi spojená.