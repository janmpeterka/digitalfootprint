\chapter{Praktická část}

\section*{Úvod}
Praktická část této balakářské práce se zabývá navržením a vytvořením prostředí (dále \textit{Aplikace}) pro vzdělávání této tématiky.

\section{Východiska}
Jak vyplývá z~kapitoly \textit{Zařazení problematiky digitální stopy do školních výukových materiálů}, téma digitální stopy a osobních dat se nejspíše bude po revizích více objevovat v~RVP středních škol. Na tuto skupinu studentů tedy bude Aplikace cílit.

Prostředí může sloužit jako součást většího vzdělávacího oblouku, ovšem tato práce si nedává za cíl vytvoření metodik pro učitele, a tedy musí obsah Aplikace fungovat i sám o~sobě.

Cílem je uživatele (studenty) v~Aplikaci seznámit s~tématem interaktivní, lákavou formou.

Zároveň však součástí Aplikace musí být kvalitní napojení na teorii i reálné příklady, včetně informování uživatele o~tom, jak může svou digitální stopu spravovat.

\subsection{Návrh aplikace z~pohledu uživatele}
Návrh aplikace je následující:
Uživatel má možnost odehrát jednotlivé \uv{Mise}, které představují fiktivní příběh. V~každé Misi má uživatel ze simulovaných osobních dat různého typu vyřešit na začátku danou otázku.
Zjednodušený use-case tedy je:
\begin{itemize}
	\item uživatel si vybírá Misi
	\item uživatel je seznámen s~příběhem Mise (symbolickým rámcem) a cílem - co je potřeba zjistit pro splnění Mise
	\item uživatel využívá dané datasety pro nalezení řešení
	\item po správném zadání odpovědi jsou uživateli zobrazeny doplňující informace - napojení na teorii, příklady z~reálného světa, možnosti zabezpečení se v~podobné situaci.
\end{itemize}

V~sekci \textit{Scénáře} budou popsány konkrétní možné scénáře, z~nich část bude použita v~prototypu.

\subsection{Prototyp}
Jak bylo řečeno, cílem práce je vytvoření prototypu. Ten tedy nemusí mít všechny funkce či grafické řešení aplikace, která by byla reálně veřejně použita ve vzdělávání, má za cíl pouze najít vhodnou podobu, otestovat její technickou proveditelnost a náročnost, a získat zpětnou vazbu od vybraných testerů.

Prototyp bude obsahovat několik scénářů s~různými typy dat.


\subsection{Scénáře}
Jak bylo popsáno, jádrem Aplikace jsou Mise - tedy jednotlivé příběhy, ve kterých se uživatel seznamuje s~různými situacemi, týkajícími se osobních dat.

Aplikace bude navržena tak, aby bylo možné snadno další scénáře přidávat, a tím rozšiřovat její vzdělávací potenciál (například v~zacílení na jiné věkové skupiny).

Při tvorbě scénářů vycházím z~dat a rizik popsaných v~předchozích kapitolách a navrhuji následující testovací scénáře, které představují různé typy a kategorie dat.

\textit{Uvědomuji si, že některé scénáře ukazují činnost, jež je ilegální. Domnívám se, že možnost prožít si situaci z~pohledu útočníka může vést k~lepšímu prožití a přenesení do uvažování o~vlastní ochraně. Zároveň u~každého takového scénáře bude upozorněno, že jde o~simulaci a jaké trestněprávní dopady by taková činnost měla v~reálném světě.}

\subsubsection*{Odhadnutí hesla}
\textbf{Cíl}\\
Uhodnout heslo blízké osoby na sociální síť.\\
\textbf{Používané zdroje dat}\\
Příspěvky na sociální síti.\\
\textbf{Typ dat}\\
Veřejné / veřejné pro okruh lidí.\\
\textbf{Dodatečné informace}\\
Informace o~tom, jak lidé tvoří hesla\\
Případy toho, kdy lidé měli veřejně informace, jež vedly k~prolomení hesla.\\
Obecná doporučení, jak chránit svoje hesla\\
Upozornění na nelegálnost takové činnosti v~reálním životě\\

V~tomto scénáři je úkolem uhodnout heslo do sociální sítě. Scénář se odkazuje na témata vhodného zabezpečení svých účtů a sicuací, kdy člověkem sdílené informace napomáhají k~prolomení jeho obran.

Jde o~jednoduchý, začáteční scénář -- pracuje s~veřejnými daty z~jednoho zdroje. Je cíleně zjednodušený oproti realitě.

\textit{Scénář se nezabývá hesly z~pohledu kryptografického a obecně bezpečnostního, neboť to je již mimo rozsah naší práce. Bylo by však pravděpodobně možné takový scénář do Aplikace přidat pro vzdělávání v~této oblasti.}


\subsubsection*{Plánování vloupání}
\textbf{Cíl}\\
Naplánovat vloupání dané osoby -- udělat si představu o~jejím bydlišti a době, kdy bude dům prázdný.
\textbf{Používané zdroje dat}\\
Příspěvky na sociální síti
Příspěvky z~fitness sociální sítě\\
Stránky firmy\\
\textbf{Typ dat}\\
Veřejná data\\
\textbf{Dodatečné informace}\\
Případy z~praxe\\
Obecná doporučení, jak se tomuto typu útoku bránit\\ 
Upozornění na nelegálnost takové činnosti v~reálním životě\\

Tento scénář již ukazuje práci s~více zdroji dat a klade na uživatele větší nároky v~hledání částí informací.


\subsubsection*{Vyšetřování korupce}
\textbf{Cíl}\\
U~podezřelé osoby vyšetřit možné korupční vazby na jiné osoby.\\
\textbf{Používané zdroje dat}\\
Lokační data od operátorů\\
Výpisy z~bankovního účtu\\
Výpisy hovorů a SMS zpráv\\
\textbf{Typ dat}\\
Soukromé -- dostupné pouze poskytovatelům a v~oprávněných případech policii\\
\textbf{Dodatečné informace}\\
Informace o~sběru a uchovávání dat operátory\\
Informace o~přesnosti lokačních dat\\
Informace o~právních možnostech policie si tato data vyžádat\\


\subsubsection*{Výběr vhodné reklamy} 
\textbf{Cíl}\\
Vybrat, jaké reklamy zobrazit jakým uživatelům.\\
\textbf{Používané zdroje dat}\\
Příspěvky na sociálních sítích\\
Chování na sociálních sítích\\
Historie prohlížení\\
Lokační data\\
\textbf{Typ dat}\\
Soukromé -- data sbírají aplikace.\\
\textbf{Dodatečné informace}\\
Napojení na teorii \textit{attention economy}\\
Informace o~možnostech nastavení ochrany soukromí v~různých aplikacích.\\

Tento scénář simuluje fungování vyhodnocování dat algoritmy firem jako je Facebook nebo Google, a následnou reklamní aukci. Má potenciál, aby na něj bylo navázáno šířeji tématem personalizované reklamy a \textit{attention economy}. 

\section{Uživatelské workflow a use-cases}

\section{Technické řešení}
\subsection{Výběr technologií}
Prototyp jsem se rozhodl vytvářet ve formě webové aplikace.
Hlavními důvody pro toto rozhodnutí bylo:
\begin{itemize}
	\item dostupnost pro uživatele i testery\\
	Není potřeba nic instalovat, stačí jakýkoli moderní webový prohlížeč.
	\item snadné vytvoření prototypu
	\item snadná rozšiřitelnost a kooperace na projektu\\ Jako open-source projekt předpokládá možnosti další spolupráce, například s~designéry. Úprava vizuálu je v~případě webové aplikace čistě HTML+CSS(+JS), což je rozšířená dovednost, oproti jiným grafickým prostředím (jako Qt či herní enginy).
\end{itemize}

Konkrétně jsem se rozhodl pro využití webového frameworku \textbf{Flask} (založeného na jazyce Python). Motivací byla osobní zkušenost s~tímto frameworkem, takže jsem měl dostatečnou představu o~tom, že je v~něm možné prototyp bez problémů vytvořit.

Na frontendové straně jsem se pro potřeby prototypu rozhodl nezavádět žádný rozsáhlý framework a pro někde nutné části Javascriptu jsem použil pouze microframework \textbf{stimulus.js} a \textbf{jQuery}.

Vývoj je verzován systémem \textbf{git} za použití veřejného repozitáře na službě GitHub.

\subsection{Základní funkcionality}
\subsubsection*{Nahrávání předem definovaných dat Mise}
	\textbf{Východiska}\\
	Mise musí být schopné používat předem definovaná data. Tato data mohou být v~různé formě. Ve fázi prototypu vytvářím data já, formu mám tedy plně pod kontrolou. V~budoucnu by aplikace mohla nabízet možnosti, jak přidávat další data misí.
	
	\textbf{Řešení}\\
	Zvažoval jsem několik řešení:
	
	\begin{itemize}
		\item data v~databázi -- například SQLite (která může být oproti jiným přímo součástí projektu)
		\item data přímo v~kódu
		\item data v~externím strukturovaném souboru
	\end{itemize}

	Variantu databáze jsem pro prototyp vyřadil z~důvodu složitější úpravy a nahlížení dat. \\
	Z~dlouhodobého hlediska mi přijde vhodná forma strukturovaného dokumentu (např. JSON) -- má výhody ve snadném přidávání dalších scénářů, a jasného oddělení dat a funkcionality (kódu). Vyžaduje však vytvoření kódu pro převod těchto dat do objektové struktury, kterou používá samotná aplikace. Proto jsem tuto variantu zařadil pro prototyp jako \textit{nice-to-have}, a v~první verzi prototypu data vytvářel přímo v~kódu v~kontroleru mise.

	V~případě velkého objemu dat by forma nahrávání s~externího souboru nemusela být nejvhodnější, a bude možná nutné přejít na variantu s~databází, či jinou. Díky použití ORM knihovny SQLAlchemy, a abstrahování logiky do modelové/objektové vrstvy, to ovšem nebude znamenat velký zásah do kódu. 

\subsubsection*{Zobrazení misi s~jednotlivými záložkami}
	\textbf{Východiska}\\
	V~náhledu mise je třeba uživateli zobrazit následující obsah:
	
	\begin{itemize}
		\item Úvod -- představení mise
		\item Data -- několik různých typů dat
		\item Místo pro řešení -- stránku nebo stránky, kde uživatel řeší daný úkol - zadává informaci (např. heslo), vybírá z~uvedených možností a podobně
		\item Závěr -- stránku s~přehledem použitých zdrojů a doplňujících informací. 
	\end{itemize}	

	\textbf{Řešení}\\
	V~prototypu jsem zobrazování vyřešil nahráním všech datových zdrojů při načtení stránky a zobrazování/skrývání pomocí jednoduchého javascriptového (stimulus.js) kontroleru. Kontroler funguje obecně pro libovolný počet datových zdrojů.
	Toto řešení má oproti jiným variantám jednoduchý kód (přehledná šablona a malý přehledný javascriptový kontroler).
	Potenciální riziko je v~tom, že pokud budou datové zdroje rozsáhlejší, bude první nahrání stránky pomalé. V~takové situaci by bylo vhodné data nenahrávat všechna při prvotním načtení stránky, ale přidat asynchronní získání dat pomocí Fetch API. To se však nijak nevylučuje s~vytvořeným kontrolerem, který má na starost pouze přepínání viditelnosti jednotlivých částí stránky.

\subsubsection*{Zobrazovat data různých typů}
	\textbf{Východiska}\\
	Definoval jsem si základní typy dat, které se v~náhledech misí mohou objevovat (dle kapitoly \textit{Hlavní zdroje dat digitální stopy uživatele}): 

	\begin{itemize}
		\item Příspěvky na sociálních sítích
		\item Výměna zpráv mezi dvěma osobami
		\item Výpis dat\\
			seznam hovorů, výpis z~bankovního účtu,...
		\item Datové body na mapě
		\item Souhrnné informace\\
			například informace, které uchovávají sociální sítě o~jednotlivých uživatelích
		\item webové stránky
	\end{itemize}

	Kromě toho ale lze očekávat, že se budou v~budoucnu objevovat další typy dat, je tedy důležité, aby bylo snadné tuto nabídku rozšiřovat, a aby bylo řešení dostatečně obecné (nebo zobecnitelné), aby bylo toto přidávání snadné. 

	\textbf{Řešení}\\
	Pro jednotlivé typy dat jsem vytvořil samostatné modely, kontrolery a sadu šablon. Pro ukázku popíši strukturu kódu pro zobrazování dat typu sociální síť:\\
	Kontroler umožňuje zobrazovat několik základních stránek - osobní profil, sadu příspěvků (feed) a přihlašovací stránku.
	Zobrazování sady příspěvků je realizováno pomocí několik \textit{partial} šablon - feed, post, comment. Jednotlivé šablony je pak snadné například designově upravovat, a zůstávají velice přehledné.

	U~většiny typů dat bylo třeba si vytvořit vlastní HTML+CSS kód, který nabízí očekávanou strukturu dat v~přehledné podobě.
	Pro zobrazování lokačních dat jsem využil SMapy pomocí javascriptu.   


\subsection{Další funkcionality}
Mezi funkcionality, které nejsou potřebné pro fázi prototypu, ale je dobré na ně myslet při strukturování celé aplikace, patří:

\begin{itemize}
	\item \textbf{Uživatelský účet}\\
		Aplikace by měla nabízet uchovávání informací o~stavu jednotlivých misí, aby mohl uživatel navázat tam, kde přestal.
		Může to být řešeno uživatelským účtem s~registrací, nebo identifikací pomocí cookies.

	\item \textbf{Přidávání dalších scénářů}\\
		Aplikace do budoucna počítá s~možností vlastních scénářů. Systém zadávání zdrojových dat a jejich nahrávání tedy musí být možné vystavit ven.
		Varianty přidávání můžou být různě technicky složité, to bude záležet na další analýze používání aplikace v~praxi.		


\end{itemize} 

\subsection{Návrh aplikace}
Aplikace má architekturu MVT (Model-View-Template), inspirovanou návrhem jiného webového frameworku v~jazyce Python, a to Django.

\subsubsection*{Model}
Aplikace plně využívá objektový návrh, a se všemi zdrojovými daty nakládá jako s~objekty.\\
Jsou tedy definované modely/třídy jako \verb|Facebook user|, \verb|Facebook post|, \verb|Location point| a další.\\
Kromě toho je objektem i každá Mise, která obsahuje jednotlivé \verb|Mission Items|.

\subsubsection*{View}
Vrstva View řeší získávání a zpracování dat, vykreslení šablony s~těmito daty a navázání na konkrétní cesty (\textit{route}).\\
V~aplikaci je použito rozšíření \verb|Flask-Classful|, které usnadňuje práci s~views, například snadným vytvořením základních \verb|CRUD| operací.

\subsubsection*{Template}
Flask v~základu využívá templatovací jazyk \verb|Jinja2|, který nabízí omezené množství logiky v~šabloně, a vede k~tomu, aby většina aplikační logiky zůstávala mimo šablonu (tedy ve View).

\subsection{Ověření a otestování aplikace}
Prototyp aplikace je třeba uživatelsky otestovat, 


\subsection{Vyhodnocení provedeného ověření a doporučení pro další rozvoj aplikace}