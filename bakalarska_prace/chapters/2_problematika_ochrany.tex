\chapter{Problematika ochrany digitálních osobních dat uživatelů}

\section{Úvod}
Rozvoj digitální části života s~sebou mimo jiné přinesl čím dál větší objem digitálních (a tedy relativně snadno uchovávatelných, duplikovatelných a strojově či částečně strojově zpracovatelných) dat. V~online světě trávíme čím dál větší množství času (kolem 7 hodin denně v~roce 2020 \citep{digital-2021-report}), a spolu s~tím jsou čím dál pokročilejší technologie pro sběr a vyhodnocování dat.

V~této kapitole se podíváme na důvody, které vedou k~aktuálnímu stavu v~oblasti sběru osobních dat, jak z~pohledu ekonomické a jiné motivace, tak s~pohledu technologického. Dále rozebereme rizika tohoto stavu.

\section{Motivace pro sběr dat}

V~diskuzích o~osobních datech a jejich sběru se nejčastěji objevují jména dvou digitálních firem - Google a Facebook (případně ještě Apple). Důvodem je samozřejmě to, že jsou obě velkou součástí nabídky digitálních služeb.

Facebook se blíží ke třem miliardám aktivních uživatelů\citep{facebook-active-users}, u~Google je výpočet složitější (protože jde o~různé služby), ale jen mezi vyhledávači (tedy esenciální služba pro většinu uživatelů internetu)'má podíl [přes 90\citep{google-search}. S~celkovým odhadem početu digitálních uživatelů 4.6 miliardy\citep{digital-2021-report} tedy jde o~více než 4 miliardy uživatelů.

Obě tyto firmy mají svůj obchodní model založený na nabízení reklamy - v~případě obou firem jde o~primární zdroj příjmů. Jejich dominance na trhu (v~USA má Google ~30\% trhu reklam\citep{google-ads} v~digitálním prostředí, Facebook následuje s ~20\%) je umožněna dobrým zacílením reklamy. Tato personalizace je umožněna právě sběrem dat a jejich analýzou.

% [https://abc.xyz/investor/static/pdf/2020Q4_alphabet_earnings_release.pdf](https://abc.xyz/investor/static/pdf/2020Q4_alphabet_earnings_release.pdf)

% [https://investor.fb.com/investor-news/press-release-details/2021/Facebook-Reports-Fourth-Quarter-and-Full-Year-2020-Results/default.aspx](https://investor.fb.com/investor-news/press-release-details/2021/Facebook-Reports-Fourth-Quarter-and-Full-Year-2020-Results/default.aspx)

\section{Technické možnosti sběru dat}

Pro tvorbu profilů jednotlivých uživatelů webu a nabízení reklamy je potřeba sbírat co nejvíce informací o~chování uživatele v~online světě. Vytvoření profilu člověka nebylo technicky téměř možné před začátkem používání \textit{cookies}.
Cookies byly poprvé implementovány v~roce 1994 do prohlížeče Netscape jako řešení pro ukládání stavu nákupního košíku \citep{cookies-history}. Cookies umožňují uchovávat informace o~uživateli na dané stránce, s~rozvojem webu a zejména vkládání skriptů a iframů do webů to umožnilo sledovat uživatele skrze cookies napříč weby, což umožnilo sledování historie prohlížení části webů a remarketing - nabízení produktů z~obchodu, který uživatel na webu navštívil\citep{scott-cookies}. To byl dlouho hlavní způsob profilování uživatelů pro reklamní účely, postupně byly možnosti cookies právně limitovány, a začaly se používat další techniky - \textit{device} a \textit{browser fingerprinting}, \textit{tracking pixels} nebo \textit{behavioral profiling}.

Na jednotlivé technologie jsou více popsány v~kapitole Klíčové zásady uživatelské ochrany osobních dat.

\section{Povědomí o problematice}

Důležitou otázkou je, jaké je aktuálně povědomí lidí o této realitě. To není snadná otázka, i z toho důvodu, že se oblast rychle mění a vyvíjí. Ovšem nějakou představu si udělat můžeme.

Podle [PEW Research](https://www.pewresearch.org/internet/2019/11/15/americans-and-privacy-concerned-confused-and-feeling-lack-of-control-over-their-personal-information/) je 59% dotázaných občanů USA podle svého mínění nedostatečně informováno o tom, jak s daty společnosti nakládá, a zároveň má 81% pocit nedostatečné možnosti kontroly, a shodné procento vnímá rizika s tím spojená z vyšší než benefity.

[Studie](https://www.sciencedirect.com/science/article/pii/S1071581920301002) zkoumající povědomí a změny ve vnímání digitálního soukromí mezi studenty po medializované kauze firmy Cambridge Analytica, dochází k závěrům, že:

- lidé málo vnímají dopady související s *networked privacy (lze přeložit jako "propojené soukromí")* - tedy uvědomění možností získávání a agregace dat o osobně z dat jiných osob.
- lidé si myslí, že jsou imunní vůči cílené reklamě, a tedy nemají obavu z toho, že budou data o nich využita k ovlivnění jich samotných
- z toho plynoucí závěr, že lidé nemohou dělat opravdu uvědomělé závěry, pokud mají takto nepřesnou představu

Dalším důležitým faktorem může být *privacy fatigue (do češtiny lze zhruba přeložit jako "únava z tématu soukromí*) - jev, který popisuje například tato [studie](https://www.sciencedirect.com/science/article/pii/S0747563217306817?via=ihub). Ta plyne z pocitu ztráty kontroly, nepřehlednosti celého tématu a pocitu marnosti z dalších a dalších úniků dat a prolomení soukromí. Studie ukazuje, že tato únava vede k menšímu zájmu o téma (*behavioral disengagement)* a menší ochotě až úplné rezignaci na změnu vlastního chování.
