\chapter*{Úvod}
\addcontentsline{toc}{chapter}{Úvod}
\pagestyle{plain}

Cambridge Analytica, Čínský kreditový systém nebo předvolání ředitelů firem jako je Facebook a Google před americký Kongres. To je několik mediálně známých případů, které v~posledních letech vynáší téma osobních dat, jejich směru a s~tím spojených rizik, do povědomí širší veřejnosti.
Jsou však uživatelé (tedy my všichni) o~tématu dostatečně informováni? Jsou tyto informace takové, abychom se zvládali v~tématu orientovat, a zároveň dělat zodpovědná, informovaná rozhodnutí o~našem vlastním chování v~digitálním světě?

Tato práce si dává za cíl téma osobních údajů a digitální stopy zmapovat, a to primárně z~pohledu jednotlivce - uživatele. Pokouší se shrnout základní témata a pojmy, které pomáhají se v~tématu zorientovat a mluvit o~něm, prozkoumat typy dat a digitální stopy, která naším každodenním chováním vzniká, upozornit na rizika, a zároveň nabídnout konkrétní přístupy, kterými já jako uživatel můžu tuto realitu ovlivňovat.

Tyto teoretické poznatky jsou pak kromě jejich samotného přínosu i podkladem pro praktickou část práce - vytvoření prostředí, ve kterém se uživatel může s~tématem seznámit. 
