\chapter*{Úvod}
\addcontentsline{toc}{chapter}{Úvod}
\pagestyle{plain}

Cambridge Analytica, Systém sociálního kreditu v~Číně nebo předvolání ředitelů firem jako je Facebook a Google před americký Kongres. To je několik mediálně známých případů, které v~posledních letech vynesly téma osobních dat, jeho vývoje a s~tím spojených rizik do povědomí širší veřejnosti.
Jsou však uživatelé (tedy všichni, kdo používají digitální služby) o~tématu dostatečně informováni? Jsou tyto informace takové, aby se lidé zvládali v~tématu orientovat a zároveň dělat zodpovědná, informovaná rozhodnutí o~svém vlastním chování v~digitálním světě?

Tato práce si klade za cíl téma osobních údajů a digitální stopy zmapovat, a to primárně z~pohledu jednotlivce -- uživatele. Pokouší se shrnout základní témata a pojmy, které pomáhají se v~tématu zorientovat a mluvit o~něm, prozkoumat typy dat a digitální stopy, která naším každodenním chováním vzniká, upozornit na rizika a zároveň nabídnout konkrétní přístupy, kterými uživatel může tuto realitu ovlivňovat.

Tyto teoretické poznatky jsou pak kromě jejich samotného přínosu i podkladem pro praktickou část práce -- vytvoření vzdělávacího prostředí, ve kterém je uživatel tímto tématem provázen.
