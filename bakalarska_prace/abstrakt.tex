\vbox to 0.5\vsize{
\setlength\parindent{0mm}
\setlength\parskip{5mm}

{\Large\bfseries Abstrakt}

Cílem bakalářské práce je zmapovat téma vytváření digitální stopy a osobních dat v~digitálním prostředí.
Práce se na tuto problematiku dívá zejména z~pohledu jednotlivce.
Zaměřuje se na to, jaký vliv může mít digitální stopa v~životě jednotlivce, a jaké možnosti kontroly nad digitální stopou jednotlivec má.

Dále se práce zabývá napojením tohoto tématu do stávajících i vznikajících rámcových vzdělávacích plánů a obecně do aktuálního školského vzdělávání.

Ze zjištěných poznatků pak vychází praktická část práce, která si klade za cíl vytvořit prototyp online prostředí pro seznámení lidí (zejména žáků a studentů) s~tímto tématem popularizační a interaktivní formou.

\vspace{4mm}

{\Large\bfseries Klíčová slova}

sběr digitálních dat, digitální stopa, bezpečnost

\vss}\nobreak\vbox to 0.49\vsize{
\setlength\parindent{0mm}
\setlength\parskip{5mm}

{\Large\bfseries Abstract}

The aim of the bachelor thesis is to map the topic of creating a digital footprint and personal data in a digital environment, and to show this issue especially from the perspective of the individual, the possible influence on him and the possibility of his influence on creating his own digital footprint.

Furthermore, the work deals with the connection of this topic to existing and emerging educational plans and in general to the current school education.

The practical part of the work, based on these findings, aims to create a prototype online environment for acquainting people (and especially pupils and students) with this topic in a popularizing and interactive form.

\vspace{4mm}

{\Large\bfseries Keywords}

collection of personal data, digital environment, security

\vss}

\newpage

